\documentclass[a4paper,11pt]{report}
\usepackage{graphicx} % Required for inserting images
\usepackage[UKenglish]{datetime}
\usepackage{subcaption}
\usepackage[pdfusetitle]{hyperref}

\newdateformat{mmmmyyyy}{\monthname[\THEMONTH]\:\THEYEAR}
\newdateformat{normaldate}{\THEDAY\:\monthname[\THEMONTH]\:\THEYEAR}

\newcommand{\comparison}[5] {
    \begin{figure}[htbp]
        \centering
        
        \begin{subfigure}[t]{0.45\textwidth}
            \centering
            \includegraphics[width=\linewidth]{#1}
            \caption{#2}
        \end{subfigure}
        \hfill
        \begin{subfigure}[t]{0.45\textwidth}
            \centering
            \includegraphics[width=\linewidth]{#3}
            \caption{#4}
        \end{subfigure}
    
        \caption{#5}
    \end{figure}
}

\title{Location and Identity Resolution from Minimal Public Signals: An Ethical OSINT Case Study}
\author{Matthew Beardwell}
\date{Revised \normaldate\today}

\begin{document}

\maketitle

\begin{center}
    \textbf{Abstract}
\end{center}

This case study explores investigative techniques for reconstructing location and identity from extremely limited input --- sparse, unstructured video publications on YouTube. It was conducted to support my professional portfolio demonstrating capability in low-signal investigative techniques, geolocation, and ethical open-source investigation relevant to cybersecurity and threat intelligence. The objective was to assess the viability of investigation methods under information poverty.

\newpage

\begin{center}
    \textbf{Ethical Declaration}
\end{center}  

This investigation was conducted solely for research and professional development in performing investigations. No private systems, credentials, or restricted data sources were accessed. All analysis was performed using publicly available information at the time of research. No contact was made with the subject or their associates. The target was treated as an anonymous case and all findings are now anonymised and disclosed only for educational purposes.

\newpage

\tableofcontents

\chapter{Introduction}

This case study demonstrates how analysis of publicly-accessible information --- including video footage, social metadata, and satellite imagery --- can resolve location and identity with very low-signal input.

In the UK, the investigative framework I will use is called `I3' --- `Intelligence, Investigation, Inference' --- which follows this information pipeline:

\vspace{0.5cm}
\begin{center}
Raw Data $\rightarrow$ Processed Data $\rightarrow$ (Investigation) $\rightarrow$ Intelligence Report
\end{center}
\vspace{0.5cm}

Processed raw public data --- structured, validated, and analysed --- is often referred to as `OSINT'. Despite the name, OSINT is not intelligence in the decision-ready information sense. It is the input to the investigation, not the output.

In this case study I use this framework to reveal locations and the identity of the subject's pseudonymous YouTube profile. The profile I chose for this purpose was selected for having minimal indicators.

I will list the film shoot locations in the order that I have identified them along with the video and its timestamp of when the location was filmed. I will explain with each one how I obtained precise GPS coordinates.

This report is anonymised --- locations, video titles, video timestamps, personal details and any other identifying information have not been included. I have transformed the original images using `posterisation' to make it more difficult to use them to re-identify their source.

\chapter{Methodology}

In this chapter, I will illustrate the investigative process from beginning to end showing how I narrowed down several film shoot locations and the subject's residence as well as how I narrowed down the subject's identity and personal metadata.

\section{Identifying Film Locations and Narrowing the Subject's Residence}

\subsection{Video A}

\subsubsection{Farmers' Market $\sim$08:10}

\comparison
    {images/grocery-youtube.png}
    {Posterised frame from Video A showing the storefront layout.}
    {images/grocery-streetview.png}
    {Posterised public image confirming feature match.}
    {Comparison of video frame and public image.}
    
Reverse image searching a video frame through Yandex.com provided many results, one of which I could immediately identify to be this same building, referred to as their “local” farmers’ market in the video. This provided a foothold to begin resolving other locations in the video content.

\subsubsection{Pet Shop $\sim$10:40}

Given the footage was filmed in a pet shop, I searched for pet shops near the rental property address I found earlier in Google Maps. Enumerating them all, I found one with user-submitted photos of the inside that matched the features in the video.

\subsection{Video B}

\subsubsection{Canyon Trail $\sim$12:00}

\comparison
    {images/canyon-video.png}
    {Posterised frame from Video B showing the town from the mountain.}
    {images/canyon-google.png}
    {Posterised public image confirming feature match.}
    {Comparison of video frame and public image.}

After extracting a video frame looking down on a town from up on a mountain, I opened an image editing tool and highlighted key features such as roads and buildings.

Now knowing the location where Video A was filmed, I opened Google Earth and positioned the view at, and pointing to, that town. I visually identified those features in the video frame from Video B in the software and aligned the view to point at them such that the view angle and position would make what I saw similar to the extracted video frame. This helped me identify a rough location it was filmed from the Google Earth's view position.

Given the steepness of the mountainsides, I guessed that they would have taken a canyon trail to reach that position. I searched for the canyon trail that passed through the Google Earth's view position and opened it on Google Street View at the view position's location. I compared a Google user-submitted photo at that position to the video frame side-by-side in an image editing tool and the key features in the images and their perspectives matched which confirmed the video frame's location.

\subsection{Video C}

\subsubsection{University-owned Housing Complex $\sim$00:40}

The footage shows the subject walking on a pavement then it cuts to them standing outside a flat complex. I searched Google Images for "apartments in [the town from previous queries]" and found a rental property listing for an address that visually matched the outside of the complex in the video. The bench's location outside the flats differed so I couldn’t be completely confident on the match however. Further research found three other similar housing complexes with a shared garden area like this in the area so I mentally noted that the footage could've also been filmed at one of the others.

\subsubsection{Softball Field $\sim$00:30}

\comparison
    {images/pavement-video.png}
    {Posterised frame from Video C showing subject walking near the complex.}
    {images/pavement-drone.png}
    {Posterised drone footage frame as it flies through the same location.}
    {Comparison of video frame and drone footage.}

Now that I have the location for the more easily-identifiable scene just seconds later, I can come back to the scene of the subject walking on the pavement. As Google Street View's coverage of this area was insufficient, I searched YouTube for videos shot near the location of the housing complex in the later scene. I came across some drone footage and followed the drone's path, identifying buildings it flew past by comparing them to buildings on Google Street View. With the drone's flight trajectory mapped, I could identify the location where is flies past the same portion of pavement in Video C.

\section{Narrowing Residence Cluster from Prior Observations}

\subsection{Inferring the Correct Flat Complex}

In Video C, the transition from the scene of the subject walking along a pavement outside their flat complex at $\sim$00:30 to in the shared garden area under their balcony at $\sim$00:40 suggests they were walking home. Using mapping tools, I notated the direction of travel and with two maps of the cluster of all four flat complexes of the same type that I had found online, it could be inferred that they lived in the northern-most complex.

\subsection{Narrowing It Down to Just Four Flats}

\comparison
    {images/garden-balcony.png}
    {Posterised frame from Video C showing the view of the garden area from the flat's balcony.}
    {images/garden-satellite.png}
    {Posterised satellite photo showing the inferred flat complex.}
    {Comparison of video frame and satellite imagery.}

In Video C, there is a shot of the subject in the shared garden area from above looking down from a balcony of one of the flats. Matching the features such as trees, benches, and the playground to a satellite photo on Google Maps, I could infer that the balcony was located on the right half of the southern-most block of flats in the complex. A brown and green bench are adjacent along with an orthogonal lighter bench. These benches roughly correspond to the same near the southern-most block however the orientations and positions slightly differ in the older satellite images.

I looked at the positions of the paths around the garden and the subject's position relative to them to identify their position on the map in the same balcony shot in Video C. This reveals roughly where the balcony is in the southern-most block of the complex.

Overlaying a university online map over the satellite image in an image editing tool with translucency, I inferred the flat numbers that corresponded to the balcony's rough location. Google Maps doesn't show the flat numbers correctly so I looked online and found a map by the university that owns the complexes. It was a bit cryptic on how the numbers corresponded to which flats so I looked on Google Street View which showed a plaque with flat numbers on them. The numbers were ascending vertically which corresponded to the same vertically ascending numbers on the online map. This let me deduce that the balcony was located outside one of eight flats --- four on floor one and four on floor two.

If further assumptions were made, you could infer a likely flat number however such assumptions weren't convincing.

\section{Resolving the Subject's Identity}

The goal of this section is to show how I went from the minimal-signal YouTube profile to a high resolution of personal metadata. I have redacted all personal information, URLs, and screenshots to protect the subject's privacy.

\subsection{Pseudonymous YouTube to Personal LinkedIn}

On the subject's earliest published video on YouTube, the description below links to a Python Jupyter Notebook file on Google Drive. The file's owner is still pseudonymous, but searching comments in this file on Google retrieves the same notebook uploaded to Google Colab by a personal email account --- the handle revealing the owner's full name.

A Google Image search for this handle reveals a portrait of a person from a LinkedIn profile. The profile's owner is wearing the same university jumper as the subject in one of their videos. The subject's STEM-related YouTube account agrees with the claim that the LinkedIn profile studied a STEM subject. The profile photo is also consistent with the subject's face from the videos.

\subsection{Personal LinkedIn to Personal Facebook}

The LinkedIn profile didn't reveal much about the subject. It showed their face and their job history, but the detail isn't much better than the YouTube profile. Using the subject's name from the LinkedIn account, I searched Facebook. There are 43 accounts under that name, one of which stood out to me as having studied at the same university as the subject. The Facebook's profile photo matches both the subject in the video as well as the LinkedIn profile photo. This Facebook account can potentially infer the subject's social network and family composition as well as the locations that they are associated with.

\subsection{Resolving More Personal Metadata}

While searching open databases, I can use the information from the social media accounts to identify which results belong to the subject. On `www.fastpeoplesearch.com', there were 15 results for the known full name. Ruling out 11 by age and another by researching the spouse, I’m left with three potential results. Only one of these has a connection to the same town as the Facebook profile. It lists a person as related whose name also appears as a commenter on the Facebook profile.

Merging results from FastPeopleSearch with what is known already, a full name, date of birth to the nearest month, personal emails, list of connections and how they are related, a previous address and a present-day address within a group of flats, and two social media profiles are found. A timeline of moving to university, marriage, graduation, moving states and starting the YouTube channel has been obtained.

\section{Tools and Techniques}

Here I will list the tools I used to carry out this investigation and data collection as well as the tools I used to create the report.

\subsection{Tools}

\begin{itemize}
    \item \textbf{Google/Yandex Reverse Image Search} for identifying locations from frames
    \item \textbf{Google Dorking} for precise control over engine searches
    \item \textbf{Google Earth and Google Maps} for visual correlation with terrain/building layouts and plaque \& flat number correlation
    \item \textbf{Google Drive / Colab} showcasing public access metadata leakage
    \item \textbf{LinkedIn \& Facebook} for open profiles
    \item \textbf{FastPeopleSearch} as a people index
    \item \textbf{ImageMagick} for posterisation, resizing, format conversion
    \item \textbf{Bash scripting} for reproducible image processing
    \item \textbf{md5sum and identify} to validate transformations and enforce consistency
\end{itemize}

\subsection{Techniques}

\begin{itemize}
    \item Frame-by-frame inference from video footage
    \item Visual feature matching across data formats (video $\leftrightarrow$ satellite $\leftrightarrow$ map)
    \item Manual geolocation using architecture and object placement
    \item Metadata pivoting (email handle to LinkedIn profile)
    \item Facial feature \& clothing correlation
    \item Ethical anonymisation through data suppression and abstraction
    \item Posterisation for privacy-preserving visual analysis
    \item Markdown-based structured documentation and visual explanation
\end{itemize}

\chapter{Conclusions}

I've demonstrated that efforts to separate online personas from real-world identities can be insufficient against an ethical, motivated, and capable actor using only publicly accessible data --- let alone a malicious one. I've learnt how powerful structured public data can be and how important it is to maintain robust ethics. It also gave me practical experience with geolocation, social graph resolution, and efficient investigative tree search under uncertainty.

This case study exemplifies how publicly available information can compromise anonymity and privacy, even in low-signal environments. It reinforces the importance of security auditing, minimising metadata, and privacy-aware system design.

\chapter{Data Sources}

\begin{itemize}
\item YouTube video uploads
\begin{itemize}
    \item Video content and descriptions
    \item Drone footage comparisons
\end{itemize}
\item Public Colab notebooks (links contained in video upload metadata)
\item Image search engines
\begin{itemize}
    \item Yandex Reverse Image Search
    \item Google Images
\end{itemize}
\item Personal metadata
\begin{itemize}
    \item LinkedIn \& Facebook (public profiles only)
    \item FastPeopleSearch (US people index)
\end{itemize}
\item Maps (conventional, satellite, and 3D)
\begin{itemize}
    \item University-provided online housing maps
    \item Google Maps / Street View / Earth
\end{itemize}
\end{itemize}

\end{document}
